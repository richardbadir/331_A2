\documentclass[12pt]{article}
\usepackage[top=1in,bottom=1in,left=1in,right=1in]{geometry}
\usepackage{alltt}
\usepackage{array}	
\usepackage{graphicx}
\usepackage{tabularx}
\usepackage{verbatim}
\usepackage{setspace}
\usepackage{listings}
\usepackage{amssymb,amsmath, amsthm}
\usepackage{hyperref}
\usepackage{oz}
\usepackage[cc]{titlepic}
\usepackage{fancyvrb}


\title{Concordia University\\
Department of Computer Science and Software Engineering\\
\textbf{SOEN 331:\\Formal Methods for Software Engineering}\\
\ \\
\textbf{Assignment 2}
\textbf{Richard Badir 40249566}
\textbf{Valentin Gornostaev 40211600}}
\date{\today}

\begin{spacing}{1.5}

\begin{document}
\maketitle
\newpage

\textbf{Problem 2}
\begin{enumerate}
\item
\begin{verbatim}

CL-USER 1 > (defvar *airports* '((YUL.Montreal)(LCY.London_UK)(LHR.London_UK)(MIL.Milan)(SFO.San_Francisco)(SDQ.Santo_Domingo)))
*AIRPORTS*

CL-USER 2 > (print *airports*)

((YUL.MONTREAL) (LCY.LONDON_UK) (LHR.LONDON_UK) (MIL.MILAN) (SFO.SAN_FRANCISCO) (SDQ.SANTO_DOMINGO)) 
((YUL.MONTREAL) (LCY.LONDON_UK) (LHR.LONDON_UK) (MIL.MILAN) (SFO.SAN_FRANCISCO) (SDQ.SANTO_DOMINGO))

CL-USER 3 > *airports*
((YUL.MONTREAL) (LCY.LONDON_UK) (LHR.LONDON_UK) (MIL.MILAN) (SFO.SAN_FRANCISCO) (SDQ.SANTO_DOMINGO))

CL-USER 4 > 
\end{verbatim}
\item $monitored$=dom($airports$)
\item $airports$ is a variable that holds one to many mappings of objects of type AIRPORT to objects of type CITY. dom($airports)=\mathbb{P}$ AIRPORT. codomain($airports)= \mathbb{P}$ CITY. $airports = \mathbb{P}$ AIRPORT$\mapsto$ CITY
\item This should be an unordered structe, because order doesn't matter in this scenario, nor does it make sense. Looking up in an unordered structure would also be a lot faster than in an ordered one $O(1)$ vs $O(n)$. Assuming a Hashmap or any variation is not permitted, a set containing tuples of type (AIRPORT, CITY) should be used. There should be a restriction on the domain of this data structure, indicating no head of tuple can be a duplicate of another head. Lists of strict length 2 with types $\langle$AIRPORT, CITY$\rangle$ could also be used to replace the tuples.
\newpage
\item 
\begin{schema}{AirportSystem}
codes~:~\mathbb{P}~AIRPORT\\
airports : AIRPORT \rightarrow CITY \texttt{~~~~~--non-injective, surjective}\\
cities : ~\mathbb{P}~CITY\\
\where
codes = \dom airports\\
deployed = \ran airports
\end{schema}
\item
\begin{schema}{Success}
\Xi AirportSystem\\
response! : MESSAGE
\where
response!~=~'ok'\\
\end{schema}
\begin{schema}{CodeAlreadyInUse}
\Xi AirportSystem\\
code? : AIRPORT\\
response! : Message
\ST
code? \in \dom airports\\
response!~=~'Code~already~in~use'
\end{schema}

\begin{schema}{AddAirportOK}
\Delta AirportSystem\\
code? : AIRPORT\\
city? : CITY\\
\where
code? \notin \dom airports\\
airports' = airports \cup \{ code? \mapsto city? \}\\
\end{schema}
\[ AddAirport~\hat{=}~~~~(AddAirportOK\wedge Success) \oplus CodeAlreadyInUse\]
\item
\begin{verbatim}
CL-USER 1 > (defvar *airports*
  '((YUL . Montreal)
    (LCY .London_UK)
    (LHR . London_UK)
    (MIL . Milan)
    (SFO . San_Francisco)
    (SDQ . Santo_Domingo)))
*AIRPORTS*

CL-USER 2 : 1 > *airports*
((YUL . MONTREAL) (LCY .LONDON_UK) (LHR . LONDON_UK) (MIL . MILAN) (SFO . SAN_FRANCISCO) (SDQ . SANTO_DOMINGO))

CL-USER 3 : 1 > (defun addAirport (code city)
  (if (find code *airports* :key #'car)
      (print "Code already in use.")
      (progn
        (push (cons code city) *airports*)
        (print "ok"))))
ADDAIRPORT

CL-USER 4 : 1 > 
\end{verbatim}
\item
\begin{schema}{UpdateAirportOK}
\Delta AirportSystem\\
code? : AIRPORT\\
city? : CITY\\
\where
code? \in \dom airports\\
airports' = airports \oplus \{ code? \mapsto city? \}\\
\end{schema}
\begin{schema}{CodeNotInUse}
\Xi AirportSystem\\
code? : AIRPORT\\
response! : Message
\ST
code? \notin \dom airports\\
response!~=~'Code~not~in~use'
\end{schema}
\[ UpdateAirport~\hat{=}~~~~(UpdateAirportOK\wedge Success) \oplus CodeNotInUse\]
\item
$
airports =
\{
YUL \mapsto Dorval,\\
LCY  \mapsto London UK,\\
LHR \mapsto London UK,\\
MIL \mapsto Milan,\\
SFO \mapsto San Francisco,\\
SDQ \mapsto Santo Domingo\\
\}$
\item\begin{verbatim}

CL-USER 1 > (defvar *airports*
  '((YUL . Montreal)
    (LCY .London_UK)
    (LHR . London_UK)
    (MIL . Milan)
    (SFO . San_Francisco)
    (SDQ . Santo_Domingo)))
*AIRPORTS*

CL-USER 2 > (defun updateAirport (code new-city)
  (let ((existing (assoc code *airports* :test #'equal)))
    (if existing
        (progn
          (setf (cdr existing) new-city)
          (print  "ok"))
        (print "Code not in use."))))
UPDATEAIRPORT-CITY

CL-USER 3 > (defun updateAirport (code new-city)
  (let ((existing (assoc code *airports* :test #'equal)))
    (if existing
        (progn
          (setf (cdr existing) new-city)
          (print  "ok"))
        (print "Code not in use."))))
UPDATEAIRPORT

CL-USER 4 > (updateAirport 'YUL 'Dorval)

"ok" 
"ok"

CL-USER 5 > (updateAirport 'LL 'GG)

"Code not in use." 
"Code not in use."

CL-USER 6 > 
\end{verbatim}
\end{enumerate}


\textbf{Remove Everything After this line ---------------------------------------------}\\
\newpage

\section*{Part 1:  Temperature monitoring system with the Z specification}

Consider a system called 'TempMonitor' that keeps a number of sensors, where each sensor is deployed in a separate location in order to read the location's temperature. Before the system is deployed, all locations are marked on a map, and each location will be addressed by a sensor. The formal specification of the system introduces the following three types:

\[ SENSOR\_TYPE,  LOCATION\_TYPE, TEMPERATURE\_TYPE  \]

\noindent We also introduce an enumerated type $MESSAGE$ which will assume values that correspond to success and error messages.\\

\noindent Provide a formal specification in Z, with the following operations:

\begin{itemize}
	\item \texttt{DeploySensorOK}:  Places a new sensor to a unique location. You may assume that some (default) temperature is also passed as an argument.
	\item \texttt{ReadTemperatureOK}: Obtain the temperature reading from a sensor, given the sensor's location.
\end{itemize}

\noindent Provide appropriate success and error schemata to be combined with the definitions above to produce robust specifications for the following interface:

\begin{itemize}
	\item \texttt{DeploySensor},
	\item \texttt{ReadTemperature}.
\end{itemize}

\newpage

\noindent \underline{Solution}:

\begin{schema}{TempMonitor}
deployed'~:~\mathbb{P}~SENSOR\_TYPE\\
map : SENSOR\_TYPE \nrightarrow LOCATION\_TYPE \texttt{~~~~~--partial bijective}\\
read : SENSOR\_TYPE  \nrightarrow TEMPERATURE\_TYPE\\
\where
deployed = \dom map\\
deployed = \dom read
\end{schema}

\begin{schema}{DeploySensorOK}
\Delta TempMonitor\\
sensor? : SENSOR\_TYPE\\
location? : LOCATION\_TYPE\\
temperature? : TEMPERATURE\_TYPE
\where
sensor? \notin deployed\\
location? \notin \ran map\\
deployed' = deployed \cup \{ sensor? \}\\
map' = map \cup \{ sensor? \mapsto location? \}\\
read' = read \cup \{ sensor? \mapsto temperature? \}
\end{schema}


\begin{schema}{ReadTemperatureOK}
\Xi TempMonitor\\
location? : LOCATION\_TYPE\\
temperature! : TEMPERATURE\_TYPE
\where
location? \in \ran map\\
temperature! = read(map^{-1}(location?))\\
\end{schema}

\begin{schema}{Success}
\Xi TempMonitor\\
response! : MESSAGE
\where
response!~=~'ok'\\
\end{schema}



\begin{schema}{SensorAlreadyDeployed}
\Xi TempMonitor\\
sensor? : SENSOR\_TYPE\\
response! : Message
\ST
sensor? \in deployed\\
response!~=~'Sensor~deployed'
\end{schema}


\begin{schema}{LocationAlreadyCovered}
\Xi TempMonitor\\
location? : LOCATION\_TYPE\\
response! : Message
\ST
location? \in \ran map\\
response!~=~'Location~already~covered'
\end{schema}


\begin{schema}{LocationUnknown}
\Xi TempMonitor\\
location? : LOCATION\_TYPE\\
response! : Message
\ST
location? \notin \ran map\\
response!~=~'Location~not~covered'
\end{schema}

\[ DeploySensor~\hat{=}~\\
~~~(DeploySensorOK \wedge Success) \oplus (SensorAlreadyDeployed \vee LocationAlreadyCovered) \]



\[ ReadTemperature~\hat{=}~(ReadTemperatureOK \wedge Success) \oplus LocationUnknown \]



\newpage



\section*{Part 2:  A booking system with the Object Z specification}

We introduce the basic types $[Person, SeatType]$. We also introduce an enumerated type $Message$ which will assume values (feel free to define your own) that correspond to success and error messages. Consider a system to book seats for a theater play. A customer can book a single seat, and a seat can only accommodate a single customer. The booking system keeps a log of the customers that have booked a seat. The system publishes a plan of the theater and it allows customers to access it online and make a booking or cancel a booking.\\

\section*{Class \texttt{Booking}}

Define a formal specification in Object-Z for class $Bookingt$ to  support the following operations:

\begin{itemize}

\item \texttt{BookOK}: Reserves a seat for a given customer.

\item \texttt{CancelOK}:  Frees a seat for a given customer.

\end{itemize}

\noindent You will also need to provide appropriate success and error schemata to be combined with the definitions above to produce \textit{robust specifications} for the following interface:

\begin{itemize}

\item \texttt{Book}, and

\item \texttt{Cancel}.

\end{itemize}

\section*{Class \texttt{Booking2}}

Subclassify \texttt{Booking} to introduce class \texttt{Booking2} that behaves exactly like \texttt{Booking}, while introducing the following operations:

\begin{itemize}

\item \texttt{GetNumberOfCustomers} returns the total number of customers who have made a booking.

\item \texttt{ModifyBookingOK} assigns an existing customer to a different seat. Provide any additional schema(ta) in order to extend the interface to include a robust operation \texttt{ModifyBooking}.

\end{itemize}

\noindent The extended interface will now include operations

\begin{itemize}

\item \texttt{GetNumberOfCustomers}, and
\item \texttt{ModifyBooking}.

\end{itemize}

\newpage

\noindent \underline{Solution}:

\noindent The main functionality for class \texttt{Booking} together with the success and error schemata are given below:

\begin{class}{Booking}
\also
\upharpoonright (Book, Cancel) \\
\begin{state}
booked:~\mathbb{P}~Person\\
booking : Person \pinj SeatType\\
capacity: \mathbb{N}\\
count : \mathbb{N}
\where
booked =~dom~booking\\
capacity > 0\\
count \geq 0
\end{state} \\
\ \\
\begin{init}
booked = \emptyset \\
capacity = 100\\
count = 0
\end{init} \\
\ \\
\begin{op}{BookOK}
\Delta (booking, count) \\
customer? : Person\\
seat? : SeatType
\ST
customer? \notin booked\\
seat? \notin \ran SeatType\\
count < capacity\\
booking' = booking \cup \{ customer? \mapsto seat? \}\\
count' = count + 1
\end{op}\\
\ \\
\begin{op}{CancelOK}
\Delta (booking, count) \\
customer? : Person\\
\ST
customer? \in booked\\
count > 0\\
booking' = \{ customer? \} \ndres booking\\
count' = count - 1
\end{op}\\
...\\
\end{class}

\newpage

\begin{class}{Booking /cont.}
...\\
\begin{op}{Success}
response! : Message
\where
response!~=~'ok'\\
\end{op}\\
\ \\
\begin{op}{CustomerExists}
customer? : Person\\
response! : Message
\ST
customer? \in booked\\
response!~=~'Customer~already~exists'
\end{op}\\
\ \\
\begin{op}{CustomerUnknown}
customer? : Person\\
response! : Message
\ST
customer? \notin booked\\
response!~=~'Customer~unknown'
\end{op}\\
\ \\
\begin{op}{SeatTaken}
seat? : SeatType\\
response! : Message
\ST
seat? \in ran~booking\\
response!~=~'Seat~taken'
\end{op}\\
\begin{op}{TheaterFull}
response! : Message
\ST
count = capacity\\
response!~=~'Theater~full'
\end{op}\\
\begin{op}{TheaterEmpty}
response! : Message
\ST
count = 0\\
response!~=~'Theater~empty'
\end{op}\\
\ \\
Book~\hat{=}~(BookOK \wedge Success) \oplus CustomerExists
		\oplus (SeatTaken \vee TheaterFull)\\
\ \\
Cancel~\hat{=}~(CancelOK \wedge Success) \oplus 
	(CustomerUnknown \vee TheaterEmpty)\\
\end{class}




\begin{class}{Booking2}
\also
\upharpoonright (Book, Cancel, GetNumberOfCustomers) \\
\ \\
Booking\\
\ \\
\begin{op}{GetNumberOfCustomers}
result! : \mathbb{N}
\where
result! = count\\
\end{op}\\
%\ \\
\begin{op}{ModifyBookingOK}
\Delta (booking) \\
customer? : Person\\
seat? : SeatType
\ST
customer? \in booked\\
seat? \notin ran~booking\\
booking' = booking \oplus \{ customer? \mapsto seat? \}
\end{op}\\
\ \\
ModifyBooking~\hat{=}~(ModifyBookingOK \wedge Success) \oplus (CustomerUnknown \vee SeatTaken)\\
\end{class}



\end{spacing}

\end{document}
